%导言区,用于配置格式
%指定编译器是xelatex
%!TEX program=xelatex 
\documentclass[10pt, UTF8]{ctexart}%指明文档是中文
\usepackage{hyperref}%导入包
\usepackage{graphicx}%导入图片包
\usepackage[b5paper]{geometry}%定义大小
%正文区
%begin end 用于指定环境
\begin{document}
    
    %配置开头
    \title{Latex+Vscode配置}
    \author{Dr. Li}
    \date{2020年2月20日}
    \maketitle

    %摘要
    \begin{abstract}
        这是一个关于Latex和Vscode相关联的配置文档,
        写于2020年2月20日,用于自己练手。
        \begin{figure}[htbp]%h当前位置,t顶部,b底部,p浮动
            \centering
            \includegraphics[width=0.5\textwidth]{D:/phase.png}
            % \caption{LOGO}
        \end{figure} 
    \end{abstract}

    %正文
   本文基于\LaTeX 排版学习笔记\cite{LaTex排版学习笔记}和知乎搜集资料
   \cite{知乎Ref1,知乎Ref2}编写,属于第一版。

    \section{软件环境}
    软件需要最新版的VsCode\cite{Vscode网址}和TexLive\cite{TexLive网址}。
    同时VsCode需要安装插件LaTex Workshop。
    \section{VsCode配置}
    


    
    %引用
    \begin{thebibliography}{123456}%这个123456是文献最大长度
        \bibitem{LaTex排版学习笔记} LaTex排版学习笔记,Zoho,2013,10,14
        \bibitem{知乎Ref1} \url{https://zhuanlan.zhihu.com/p/106167792?utm_source=wechat_session&utm_medium=social&utm_oi=619471395880960}
        VS CODE+LATEX 完全解决方案(2020年)
        \bibitem{知乎Ref2} \url{https://zhuanlan.zhihu.com/p/90526218}
        配置VScode编辑LaTeX及正反向搜索等设置
        \bibitem{Vscode网址}  \url{https://code.visualstudio.com}
        \bibitem{TexLive网址} \url{http://tug.org/texlive/acquire-netinstall.html}
        %使用XeTex编译时不能使用中文编辑\bibitem
    \end{thebibliography}

\end{document}

