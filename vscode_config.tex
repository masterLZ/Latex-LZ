%导言区,用于配置格式
%指定编译器是xelatex
%!TEX program=xelatex 
\documentclass[10pt, UTF8]{ctexart}%指明文档是中文
\usepackage{hyperref}%导入包
\usepackage{graphicx}%导入图片包
\usepackage[b5paper]{geometry}%定义大小
%改变字体背景颜色需要color和framed宏包
\usepackage{framed}
\usepackage{color}% 导入颜色
\usepackage{underscore}%取消下划线识别
\definecolor{shadecolor}{rgb}{0.92,0.92,0.92}
%自定义函数
\def\CodingInline#1{\colorbox{shadecolor}{\color{black}#1}}%内联代码
\def\CodingBlock#1{\begin{shaded} \noindent#1 \end{shaded}}
\def\Emphasize#1{\textcolor[rgb]{1,0,0}{#1}}
%正文区
%begin end 用于指定环境
\begin{document}
    
    %配置开头
    \title{Latex+Vscode配置}
    \author{Dr. Li}
    \date{2020年2月20日}
    \maketitle

    %摘要
    \begin{abstract}
        这是一个关于Latex和Vscode相关联的配置文档,
        写于2020年2月20日,用于自己练手。
        \begin{figure}[htbp]%h当前位置,t顶部,b底部,p浮动
            \centering
            \includegraphics[width=0.5\textwidth]{D:/phase.png}
            % \caption{LOGO}
        \end{figure} 
    \end{abstract}

    %正文
   本文基于\LaTeX 排版学习笔记\cite{LaTex排版学习笔记}和知乎搜集资料
   \cite{知乎Ref1,知乎Ref2}编写,属于第一版。
    \newpage
    \section{软件环境}
    软件需要最新版的VsCode\cite{Vscode网址}和TexLive\cite{TexLive网址}.同时VsCode需要安装插件LaTex Workshop。
    \section{VsCode配置}
    一般安装后使用默认配置就可以使用,搭配自带的Vscode tab 浏览器就可以实现文件编译,实时查看和正反向搜索。但是本着折腾一次就折腾明白的原则,叨叨一下一些基本的配置。
    \subsection{快捷键}
        只替换了编译键,其他常用键列在表格里面。
        \begin{table}[htbp]
            \centering
            \caption{快捷键}
            \begin{tabular}{|c|ccc|c|}
            
                \hline
                功能 & 原始快捷键 & \, & 修改快捷键 &备注\\
                \hline
                编译文件 & Ctrl+Alt+B & $\Rightarrow$ & Ctrl+B & \\
                \hline
                预览PDF & Ctrl+Alt+V & $\Rightarrow$ & - & -为不修改\\
                \hline
                切换编译方式 & Ctrl+r & $\Rightarrow$ & - & -为不修改\\
                \hline
                正向搜索 & Ctrl+Alt+j & $\Rightarrow$ & - & 从tex到PDF\\
                \hline        

            \end{tabular}
        \end{table}

    \subsection{setting.json设置}
        \subsubsection{通用设置}
            %设置背景颜色
            % \colorbox{red}{\color{black}红底黑字}
            % \fcolorbox{red}{green}{红框绿背景} %框色+背景色
            \begin{shaded}
                %\noindent 本段取消首航缩进
                \noindent "latex-workshop.showContextMenu":true,         //打开右键菜单
                \\"latex-workshop.intellisense.package.enabled": true,  //根据加载的包,自动完成命令或包  
                \\"latex-workshop.latex.autoBuild.run": "never",        //禁止保存文件时自动编译
                \\"latex-workshop.latex.clean.fileTypes": [  //设定清理文件的类型 
                \\"*.aux","*.bbl", "*.blg", "*.idx", "*.ind", "*.lof", "*.lot", "*.out", "*.toc", "*.acn", "*.acr", "*.alg", "*.glg", "*.glo", "*.gls", "*.ist", "*.fls", "*.log", "*.nav", *.snm", "*.synctex.gz", *.fdb_latexmk" 
                \\],  
            \end{shaded}

        \subsubsection{指定编译器}
        具体编译过程需要翻看LaTex-Workshop的英文编译文档\cite{Latex-Workshop-Compile},这里只对常用设置进行说明.        
        
        LaTeX-Workshop中,编译LaTeX文件执行顺序的命令成为LaTeX recipes。在默认情况下由\CodingInline{latex-workshop.latex.recipes}和\CodingInline{latex-workshop.latex.Tools}来定义。

        LaTeX-Workshop默认编译方式为\CodingInline{latex}和\CodingInline{padlatex}。如下方式修改了latex workshop的编译方式和辅助文件。添加xelatex和bib编译。

        打开用户设置\CodingInline{setting.json},首先设置\CodingInline{latex-workshop.latex.recipes}.
        \CodingBlock{"latex-workshop.latex.recipes": [ \\ 
            \{  \\
              "name": "latexmk",  \\
              "tools": [  \\
                "latexmk"  \\
              ]  \\
            \},  \\
            \{  \\
              "name": "PDFlatex",  \\
              "tools": [  \\
              "pdflatex"  \\
              ]  \\
            \},  \\
            \{  \\
              "name": "pdflatex ->bibtex ->pdflatex2",  \\
              "tools": [  \\
                "pdflatex", \\ 
                "bibtex",  \\
                "pdflatex",  \\
                "pdflatex"  \\
              ]  \\
            \},  \\
            \{  \\
              "name": "xelatex",\\  
              "tools": [  \\
                "xelatex"  \\
              ]  \\
            \},  \\
            \{  \\
              "name": "xelatex -> bibtex -> xelatex",  \\
              "tools": [  \\
                "xelatex",  \\
                "bibtex",  \\
                "xelatex",  \\
              ]  \\
            \}  \\
          ],  }
          接着设置\CodingInline{latex-workshop.latex.tools}
          \CodingBlock{"latex-workshop.latex.tools":[  \\
          \{  \\
            "name": "latexmk",  \\
            "command": "latexmk",  \\
            "args": [  \\
              "-synctex=1",  \\
              "-interaction=nonstopmode",  \\
              "-file-line-error",  \\
              "-pdf",  \\
              "-outdir=%OUTDIR%",  \\
              "%DOC%"  \\
            ],  \\
            "env": \{\}  \\
            \},  \\
            \{  \\
            "name": "pdflatex",  \\
            "command": "pdflatex",  \\
            "args": [  \\
              "-synctex=1",  \\
              "-interaction=nonstopmode",  \\
              "-file-line-error",  \\
              "%DOC%"  \\
            ],  \\
            "env": \{\}  \\
            \},  \\
            \{  \\
            "name": "xelatex",  \\
            "command": "xelatex",  \\
            "args": [  \\
              "-synctex=1",  \\
              "-interaction=nonstopmode",  \\
              "-file-line-error",  \\
              "%DOC%"  \\
            ],  \\
            "env": \{\}  \\
            \},  \\
            \{  \\
            "name": "bibtex",  \\
            "command": "bibtex",  \\
            "args": [  \\
              "%DOCFILE%"  \\
            ],  \\
            "env": \{\}  \\
            \}  \\
            ],  }
        \subsubsection{外部浏览器和正反向设置}
        外部浏览器主要选用SumatraPDF,主要原因是打开速度非常快,可以提供快速预览,也可以复制出任意多个文件进行任意配置.
        
        首先配置默认外部pdf阅读器,这里我的SumatraPDF的路径是D:/Program Files/SumatraPDF/SumatraPDF.exe"
        \CodingBlock{"latex-workshop.view.pdf.viewer":"external",    //pdf文件的预览方式\\
        "latex-workshop.view.pdf.ref.viewer": "external",
        //设置外部预览器\\
        "latex-workshop.view.pdf.external.viewer.command":"D:/Program Files/SumatraPDF/SumatraPDF.exe",\\
        "latex-workshop.view.pdf.external.viewer.args":[
            "\%PDF\%"
        ],}
        接着设置正向搜索这里需要一个配置文件路径,我这里是D:/Program Files/Microsoft VS Code/resources/app/out/cli.js。根据自己的\CodingInline{cli.js}路径来进行修改
        \CodingBlock{"latex-workshop.view.pdf.external.synctex.command": "D:/Program Files/SumatraPDF/SumatraPDF.exe",\\
        "latex-workshop.view.pdf.external.synctex.args": [\\
          "-forward-search",\\
          "\%TEX\%",\\
          "\%LINE\%",\\
          "-reuse-instance",\\
          "-inverse-search",\\
          "code \textbackslash"D:/Program Files/Microsoft VS Code/resources/app/out/cli.js\textbackslash" -r -g \textbackslash"\%f:\%l\textbackslash",\\
          "\%PDF\%"\\
        ],}
        接着设置反向搜索,在SumatraPDF设置选项中输入
        \CodingBlock{"C:\textbackslash Program Files\textbackslash Microsoft VS Code\textbackslash Code.exe" "C:\textbackslash Program Files\textbackslash Microsoft VS Code\textbackslash resources\textbackslash app\textbackslash out\textbackslash cli.js" -g "\%f":\%l}

        \section{结语}
        本次配置大部分采用默认配置,下次更新会更新每个参数具体含义,进行更为详细的配置。




    


    
    %引用
    \begin{thebibliography}{123456}%这个123456是文献最大长度
        \bibitem{LaTex排版学习笔记} LaTex排版学习笔记,Zoho,2013,10,14
        \bibitem{知乎Ref1} \url{https://zhuanlan.zhihu.com/p/106167792?utm_source=wechat_session&utm_medium=social&utm_oi=619471395880960}
        \newline VsCode+\LaTeX 完全解决方案(2020年)
        \bibitem{知乎Ref2} \url{https://zhuanlan.zhihu.com/p/90526218}
        \newline 配置VScode编辑\LaTeX 及正反向搜索等设置
        \bibitem{Vscode网址}  \url{https://code.visualstudio.com}
        \bibitem{TexLive网址} \url{http://tug.org/texlive/acquire-netinstall.html}
        \bibitem{Latex-Workshop-Compile} \url{https://github.com/James-Yu/LaTeX-Workshop/wiki/Compile#latex-recipes}

    \end{thebibliography}

\end{document}

